\documentclass[conference]{IEEEtran}%
\usepackage{amsfonts}
\usepackage{amsmath}
\usepackage{amssymb}
\usepackage{graphicx}%

\usepackage{color}
\definecolor{darkblue}{rgb}{0,0,0.4}
\definecolor{darkgreen}{rgb}{0,0.3,0}
\definecolor{darkred}{rgb}{0.6,0,0}

\usepackage[
pdfauthor={Jingyi Fang},
pdftitle={Artificial Centipede},
pagebackref=true,
pdftex,
pdfpagelabels,
hypertexnames=true,
plainpages=false,
naturalnames=false
]{hyperref}

\hypersetup{
bookmarksnumbered,
pdfpagemode=UseOutlines,
colorlinks,
citecolor=darkgreen,
filecolor=darkred,
linkcolor=darkblue,
urlcolor=darkblue,
breaklinks=true,
pdfborder={0 0 0}
}

\setcounter{MaxMatrixCols}{30}
%TCIDATA{OutputFilter=latex2.dll}
%TCIDATA{Version=5.50.0.2953}
%TCIDATA{CSTFile=IEEEtran.cst}
%TCIDATA{Created=Monday, March 14, 2011 23:40:29}
%TCIDATA{LastRevised=Tuesday, March 15, 2011 16:30:53}
%TCIDATA{<META NAME="GraphicsSave" CONTENT="32">}
%TCIDATA{<META NAME="SaveForMode" CONTENT="1">}
%TCIDATA{BibliographyScheme=Manual}
%TCIDATA{<META NAME="DocumentShell" CONTENT="Articles\SW\IEEE Transactions for Conferences">}
%BeginMSIPreambleData
\providecommand{\U}[1]{\protect\rule{.1in}{.1in}}
%EndMSIPreambleData
\newtheorem{theorem}{Theorem}
\newtheorem{acknowledgement}[theorem]{Acknowledgement}
\newtheorem{algorithm}[theorem]{Algorithm}
\newtheorem{axiom}[theorem]{Axiom}
\newtheorem{case}[theorem]{Case}
\newtheorem{claim}[theorem]{Claim}
\newtheorem{conclusion}[theorem]{Conclusion}
\newtheorem{condition}[theorem]{Condition}
\newtheorem{conjecture}[theorem]{Conjecture}
\newtheorem{corollary}[theorem]{Corollary}
\newtheorem{criterion}[theorem]{Criterion}
\newtheorem{definition}[theorem]{Definition}
\newtheorem{example}[theorem]{Example}
\newtheorem{exercise}[theorem]{Exercise}
\newtheorem{lemma}[theorem]{Lemma}
\newtheorem{notation}[theorem]{Notation}
\newtheorem{problem}[theorem]{Problem}
\newtheorem{proposition}[theorem]{Proposition}
\newtheorem{remark}[theorem]{Remark}
\newtheorem{solution}[theorem]{Solution}
\newtheorem{summary}[theorem]{Summary}

\DeclareMathOperator{\tr}{tr}

\begin{document}

\title{Artificial Centipede:\\
Physics, Locomotion, Perception, Behavior}
\pubid{0000--0000/00\$00.00~\copyright ~2002 IEEE}
\specialpapernotice{(Spring 2012 Written Qualifying Exam Paper)}

\setlength{\columnsep}{30pt}

\author{\authorblockN{Jingyi Fang}
\authorblockA{\\Computer Science Department\\
University of California, Los Angeles}
}%

%EndExpansion
%

%TCIMACRO{\TeXButton{Make Title}{\maketitle}}%
%BeginExpansion
\maketitle
%EndExpansion
%

\makeatletter
\def\acknowledgements{\normalfont
    \if@twocolumn
      \@IEEEabskeysecsize\bfseries\textit{Acknowledgements}---\relax
    \else
      \begin{center}\vspace{-1.78ex}\@IEEEabskeysecsize\textbf{Acknowledgements}\end{center}\quotation\@IEEEabskeysecsize
    \fi\@IEEEgobbleleadPARNLSP}
\makeatother

\begin{acknowledgements}
All the work presented in this paper was done by Jingyi Fang,
including the following innovations: (1) application of the Finite
Volume Method to biomechanical simulation in the field of artificial
life, (2) a new rigid/deformable object two-way coupling scheme, (3) a
new theory of locomotion control systems in myriapoda, and (4)
application of this theory to the design of a local, robust, leg
locomotion controller network. Professor Demetri Terzopoulos is my
advisor in the Computer Science Department. His course ``Artificial
Life for Computer Graphics and Vision'' piqued my interest to design
and implement an artificial creature, and his encouragement supported
me throughout this project. I also learned a lot about AI in
autonomous agent modeling from Professor Michael Dyer in his very
exciting ``Animat Modeling'' course. I implemented a prototype
centipede model for the course project. I thank Professor Joseph Teran
for teaching me everything I needed to know about the Finite Volume
Method in his ``Numerical Methods'' course. My sincere thanks to my
fellow PhD student Garett Ridge for carefully proofreading this paper.
\end{acknowledgements}

\smallskip

%TCIMACRO{\TeXButton{Begin abstract}{\begin{abstract}}}%
%BeginExpansion
\begin{abstract}%
%EndExpansion

This paper presents the design of a virtual centipede, including its
biomechanics, locomotion control, and simple high-level adaptive
behaviors. A computationally efficient Finite Volume Method (FVM) is
applied for the first time in the field of artificial life to simulate
the centipede's deformable body. A novel method is invented to achieve
two-way coupling between FVM soft-body dynamics and rigid-body
dynamics in a very simple and fast way. Instead of designing or
training an expensive, complex global controller network to coordinate
hundreds of legs, the virtual centipede locomotes using a robust local
controller network that can predict optimal future leg configurations
and achieve those configurations through inverse kinematics. A theory
is proposed for how the locomotion systems of myriapoda can function
robustly in a decentralized way. The locomotion of the centipede
emerges from the independent procedural behavior of each leg. This
decentralized design is computationally inexpensive yet very robust to
disturbances, and it may reflect the locomotion mechanism of natural
myriapoda. In locomotion simulation, the wave-like patterns in the
centipede's hundreds of legs emerged as a result of each leg
independently sensing and reacting to the simulated physical
environment.\\

\noindent\emph{Keywords: Computer Graphics, Artificial Life,
Autonomous Agents, Behavioral Animation, Decentralized Legged
Locomotion Control, Physics-Based Modeling, Finite Volume Method}
\end{abstract}%

\section{Introduction}

From a philosophical point of view, John Von Neumann defined life as
an abstract logical process independent of any particular medium,
including physics \cite{von}. In 1986, Christopher Langton coined the
term ``Artificial Life'' (ALife) to name of the emerging field in
which computer scientists implement simulated living systems that
emulate those found in real life \cite{wikiALife}, in part, to better
understand the principles of natural living systems.

Since natural biological entities live and evolve in an environment
dominated by Newtonian physics, it is of particular interest to
situate artificial creatures in the physical environment. The
cybernetic roboticist W.~Grey Walter constructed some of the first
autonomous robots in the late 1940s, demonstrating how simple analog
circuits in 3-wheeled tortise-like robots can give rise to phototaxis
behaviors. In his famous monograph \cite{vehicle}, Braitenberg
demonstrated more complex behaviors through generations of car-like
vehicles governed by several simple rules. Many researchers have
experimented with ALife robots like the Sony Corporation's dog-like
AIBO. Although real-world robots have proven useful in developing
ALife theories, doing artificial life research in computer-simulated
virtual environment has its advantages and is much more convenient.

During the past decades, many interesting and important works have
been done by computer scientists on simulating different creatures in
virtual physical environments. In the landmark work by Tu and
Terzopoulos on Artificial Fishes \cite{fish}, a framework using
mass-spring-damper systems for biomechanics simulation is developed to
represent the deformable fish body capable of producing muscle-based
locomotion. Such biomechanics are limited to animals lacking legs,
such as fish and snakes. Later work on salamander simulation
\cite{salamander} added rigid-body dynamics to simulate rigid legs,
plus a global artificial neural network trained to control the 4 legs
with 8 kinematic degrees-of-freedom. The resulting global controller,
often referred to as a Central Pattern Generator (CPG) is frequently
used in robotic design of hexapods \cite{hexapod} and quadrupeds
\cite{quadruped}. A well trained CPG works well and is robust against
disturbances, such that the robot can walk over irregular terrains
\cite{bigdog}. But the training time for a centralized CPG can grow
dramatically with growth in the number of degrees-of-freedom to be
trained.

The subphylum of arthropods {\it myriapoda} is a group containing
13,000 species, among them the well-known millipedes and centipedes.
Their primary characteristic is that they have tens or even hundreds
of legs spread on a linear segmented body structure, plus antennas for
sensing. During its locomotion, the wave pattern of the centipedes'
legs steadily move the entire body, even over very irregular terrain.
If we compare four-legged animals to cars, centipedes are like trains;
however, centipedes can walk over very rough terrain while a train can
only ride on a relatively straight, flat track. The unusual structure
of the centipede makes it a very interesting subject for research,
especially its locomotion system.

In this paper, I develop a state-of-the-art physical model to simulate
centipedes and design a robust decentralized local leg controller
model to achieve global locomotion via a beautiful wave-like pattern
of leg movements, plus a simple higher-level behavioral model. To
simulate the centipede's deformable body, I apply for the first time
in ALife a computationally efficient Finite Volume Method (FVM) and
propose an efficient two-way coupling between the FVM soft-body
dynamics and rigid-body dynamics. I further propose a theory for how
the locomotion systems of myriapoda can function robustly in a
decentralized way. The locomotion of the centipede emerges from the
independent procedural behavior of each leg, using a robust local
controller network that can predict optimal future leg configurations
and achieve those configurations through inverse kinematics. My
decentralized design is computationally inexpensive yet very robust to
disturbances; the wave-like patterns in the centipede's hundreds of
legs emerges as a result of each leg independently sensing and
reacting to the simulated physical environment.

\section{Overview of the Virtual Centipede}

\subsection{Functional Overview}

\begin{figure}[ptb]
\centering
\includegraphics[width=3.5in]{./images/overview.jpg}
\caption{Overview of the virtual Centipede: The green arrow shows the
flow of information in the controller network. The blue arrow shows
the physics feedback into the neural network through the centipede's
sensor neurons. Higher order behaviors are built upon basic networks
such as locomotion and sensor networks. The brain has two behavioral
repertoires; the first includes hard-coded instincts stemming from the
centipede's genes, which are evolved through generations of natural
selection and are thus the basis of centipede survival, whereas the
second includes learnt adaptive behaviors that promotes survival
throughout the lifetime of the centipede.}
\label{overview}
\end{figure}

Figure~\ref{overview} shows how the virtual centipede is constructed.
I pursued a bottom-up, compositional approach in which I modeled not
only the form and superficial appearance, but also the fundamental
physics of the centipede and its environment. In my work, the
traditional mass-spring-damper system for simulating deformable bodies
is replaced with the Finite Volume Method \cite{fvm1,fvm2} to achieve
more accurate simulation of physics. Based on this more accurate
physics model, controller networks can be designed or trained to
enable the centipede to be completely autonomous. Like real centipedes
in nature, the virtual centipede has virtual sensor neurons to sense
deformations of its body and contact, morsels of food, hormones of the
opposite sex, etc. It also has motor neurons to actuate the legs,
internal neurons to pass and synthesize information, etc. I have used
an inhomogeneous neural network \cite{randall}, which is more suitable
for the purposes of design rather than evolution. Nevertheless, it can
still incorporate traditional artificial neural networks \cite{ann}
whose weights can be trained to optimize locomotion performance.

The locomotion network serves as the foundation of the higher-level
networks, yet it is the most challenging part of this work.
Considering a moderately-sized centipede with 20 pairs of legs with
each leg having 3 degrees of freedom, the total of 120 degrees of
freedom defines a high-dimensional space in which to search for a
global control algorithm that produces optimized solution, let alone
simulate the correct behavior of those 40 legs as a real time task. To
make matters even worse, unlike the Artificial Fishes \cite{fish} that
live in an underwater environment, failure can easily occur in legged
locomotion when a couple of legs fail to cooperate correctly. Training
a CPG \cite{japan2} is an option worth trying; however, researchers in
biology \cite{walknet1,walknet2} and robotics \cite{japan,japan2,fcp}
discovered the feasibility of decentralized control systems for legged
insects or robots. Unfortunately, these efforts lack a supporting
theory that links the control system the the underlying physics, as
well as with the immanent properties of neural networks. In this
paper, I will explain how localized leg controllers can work perfectly
well for a centipede and defend the theory by experimenting with my
simple leg controller network in my virtual centipede.

The antenne of myriapoda play a vital role for sensing useful
information in the world around them. Information from antennas as
well as other sensors on the body can be synthesized by the brain to
make a high-level decision in locomotion, carried out through the
locomotion system. After a robust locomotion system is built up,
higher-level behaviors that connect perception to action, such as
obstacle avoidance, foraging, predating, etc., can be achieved either
explicitly through simple symbolic rules or through a trained implicit
neural network.

\subsection{Simulation Framework}

\begin{figure}[ptb]
\centering
\includegraphics[width=3.5in] {./images/threeTypes.png}
\caption{Different types of simulated virtual myriapoda}
\label{types}
\end{figure}

For the purposes of this project, I independently developed a small
library for the simulation and simple rendering of rigid bodies and
deformable objects. Rigid-body dynamics is implemented according to
\cite{rigid1} with impulse collision resolution described in
\cite{rigid2}. Physics-based deformable objects can be constructed
either through mass-spring-damper systems or the FVM \cite{fvm1}.
Rigid/deformable collisions are handled using penalty forces for
penetration \cite{collision,collision2}. For rendering, GLSL-based
shaders \cite{glsl} are used to optimize the performance of the GPU
drawing, such that the graphics does not act as the bottleneck in
real-time simulation. The popular open-source libraries OpenGL
\cite{opengl} and Eigen \cite{eigen} are used for the graphics and
linear algebra workloads, respectively.

My small library is used for the real-time simulation and rendering of
virtual centipedes, but it is engineered so as to be easily extensible
to constructing and rendering other types of virtual creatures. In
particular, simply by adjusting geometric parameters defining the body
structure, different types of myriapoda insects can be simulated and
they all work well using the current locomotion controller
(Figure~\ref{types}).

\section{3D Biomechanical Model}

\subsection{Overview}
\begin{figure}[ptb]
\centering \includegraphics[width=3.5in]{./images/structure.png}
\caption{Biomechanics of the virtual centipede}
\label{structure}
\end{figure}

In their 1994 paper \cite{fish}, Tu and Terzopoulos applied
mass-spring-damper systems to simulate deformable fish bodies, and
successfully applied learning algorithms to enable the fish to swim
freely underwater. The method is easy to understand and implement, so
it soon became a benchmark in artificial life research. However,
simplicity comes with several costs. Without additional volumetric
constraints, mass-spring-damper systems cannot capture volumetric
effects, such as volume conservation and the prevention of volume
inversions. The reason is that the energy of these systems is defined
over a finite number of uniaxial deformable elements rather than
volumetric elements and the biomechanics of the simulated animal body
depends on how these elements are assembled network and how their
parameters are tuned, which can be difficult. Furthermore, many
creatures have rigid skeletons and volumetric muscles that actuate
them. Although applying learning algorithms to mass-spring-damper
systems can result in desired locomotion patterns
\cite{fish,salamander}, it cannot capture the full biomechanical
realism; thus, locomotion learning results may not accurately reflect
real control mechanics evolved by nature.

Researchers have successfully attempted to use more physically
continuum mechanics models and solve the associated PDEs using Finite
Element Methods (FEM) on human skin \cite{skinfem}, flesh
\cite{fleshfem}, bone \cite{bonefem}, and muscle \cite{musclefem}
simulations. However, the complex theory and computational complexity
of the FEM has made it unpopular for real-time ALife simulations
\cite{nofem1,nofem2}. As opposed to the difficulties of the FEM, the
Finite Volume Method (FVM) is as simple to understand and implement as
mass-spring-damper systems. Like the FEM, however, the FVM is also
continuum-mechanics-based, so adequate physical realism is preserved.
Although its order of numerical accuracy \cite{fvm1} is lower than the
FEM, two aspects make the FVM very attractive in computer graphics
simulations. First, it has a firm basis in geometry, as opposed to the
FEM variational setting, which according to \cite{fvm1} not only
increases the level of intuition and simplicity of implementation, but
also increases the likelihood of aggressive (yet robust) optimization
and control. Secondly, there is a large community of researchers using
these methods to model solids subject to large deformations
\cite{fvm3}. In this paper, I will show how simple and intuitive it is
to apply the FVM in the real-time simulation of deformable centipede
body parts.

Like all myriopoda, centipedes have a rigid exoskeleton that wraps
around its soft body tissues. I modeled the centipede's body with two
types of interconnecting segments (Figure~\ref{structure}). The first
type of body segment is called the rigid body segment, and it
simulates the exoskeleton of the centipede. A pair of legs is attached
to this rigid body segment, and their pose will directly determine the
position and orientation of the rigid body segment. The second type of
body segment is called the deformable body segment and it connects
rigid segments together. The soft nature of the deformable body
segment enables turning and locomotion across irregular terrains.
Also, deformation forces can be sensed by the rigid segment and used
as an input into the leg controller (Figure~\ref{target}), which is a
target to minimize in planning future leg movements. Legs are
simulated as three rigid links that are jointed together by motors;
each leg has three degrees of freedom, enable the positioning of leg
tips or roots to a targeted position in 3 dimensional space. Inverse
kinematics is used to resolve the relative leg segment angles.

\subsection{Rigid Segment}

At first I included only deformable segments in the biomechanical
model. However, coupling the legs to the deformable segments increase
the difficulty of control, since the legs have only a point force
effect on the body. On the other hand, having the leg connect to a
rigid segment simulates the natural exoskeleton of centipedes and
enables the leg to directly control the whole segments' orientation
and position. Rigid-body dynamics have been simulated in computer
graphics since the 1980s \cite{rigid}. I implemented my own simulator
for the rigid segments of the centipede based on \cite{rigid1,rigid2}.
Impulse-based point-face and edge-edge collision detection/resolution
are also implemented such that even when the legs fail, the body of
the millipede will be supported by the ground in a physically correct
way.

\subsection{Deformable Segment}

\begin{figure}[ptb]
\centering
\includegraphics[width=3.5in] {./images/mapping.png}
\caption{Continuum Mechanics: mapping from material space to world space}%
\label{continuum}
\end{figure}

\begin{figure}[ptb]
\centering
\includegraphics[width=3.5in]{./images/FVM.png}
\caption{Gelatin Strip simulated using Finite Volume Method with collision resolution}
\label{fvm}
\end{figure}

A rigorous description of Finite Volume Method and its application in
computer graphics can be found in \cite{fvm1}. Here, I take a
practical approach in explaining the FVM in 3D. Figure~\ref{continuum}
shows the continuum mechanics view of materials in 3D. Let
$\mathbf{X}$ represent the material in its rest state (minimized
energy, zero elastic force) and let $\mathbf{x(t) =
\phi(\mathbf{X},t)}$ be the coordinate of material point $\mathbf{X}$
at time $t$ with $\phi(\mathbf{X},t)$ being the mapping function. The
strain at a material point is defined as $\mathbf{F} = {\partial
\phi}/{\partial \mathbf{x}}$. A constitutive model relates strain to
an elastic energy density function $\Psi$. The popular Neo-Hookean
constitutive model \cite{nonlinear} is nonlinear and supports large
deformations. Its energy density is given by
\begin{equation}\label{eq_neo}
    \Psi(\mathbf{F}) = \frac{\mu}{2}(\tr(\mathbf{F}^T\mathbf{F}) - 2)^2
    - \mu \ln(J) + \frac{\lambda}{2}\ln^2(J),
\end{equation}
where
\begin{equation}\label{eq_neo2}
    J = \det(\mathbf{F})
\end{equation}
and
\begin{equation}\label{lame}
    \lambda = \frac{K*\nu}{(1+\nu)*(1-2\nu)};\quad  \mu = \frac{K}{2*(1+\nu)}
\end{equation}
are the Lam\'e parameters given in terms of the Young's modulus $K$
and Poisson ratio $\nu$. I used this constitutive model to model the
millipede's soft body segment.

In the FVM, as in the FEM, the material space is partitioned into a
finite number of elements, usually tetrahedra. There are numerous
methods to do the meshing \cite{mesh1,mesh2}, but the result will not
be affected by the way the mesh is constructed. In the FVM, strain as
well as the elastic energy density can be computed individually over
the tetrahedra as follows (refer to Figure~\ref{continuum}): Since the
tetrahedron is the 3$-$simplex, the transformation matrix $\mathbf{F}$
is exactly the strain ${\partial \phi}/{\partial \mathbf{x}}$. The
matrix $\mathbf{F}$ describes the rotation and, more importantly, the
scaling of the tetrahedron. Equation (\ref{eq_neo}) shows that energy
arises from scaling (the first $\mu$ term) and resistance against
change in volume (the $\ln(J)$ terms):
\begin{equation}\label{eq_map}
\phi(\mathbf{X}) = \mathbf{F}\mathbf{X} + \mathbf{b}
\end{equation}
\begin{equation}\label{eq_map2}
\mathbf{F} = \mathbf{D}_s\mathbf{D}^{-1}_t
\end{equation}
\begin{equation}\label{eq_map3}
\mathbf{D}_s = (\mathbf{x}_4 - \mathbf{x}_1,\mathbf{x}_3 - \mathbf{x}_1,\mathbf{x}_2 - \mathbf{x}_1)
\end{equation}
\begin{equation}\label{eq_map4}
\mathbf{D}_t = (\mathbf{X}_4 - \mathbf{X}_1,\mathbf{X}_3 - \mathbf{X}_1,\mathbf{X}_2 - \mathbf{X}_1)
\end{equation}
\begin{equation}\label{eq_energy}
e(\mathbf{x}_1,\mathbf{x}_2,\mathbf{x}_3,\mathbf{x}_4) = V_0*\Psi(\mathbf{F})
\end{equation}
Given (\ref{eq_energy}), force can be derived as
\begin{equation}\label{eq_force}
\mathbf{f}_i = -\frac{\partial e}{\partial x}_i = -V_0\frac{\partial \Psi}{\partial \mathbf{F}}\frac{\partial \mathbf{F}}{\partial \mathbf{x}_i}
\end{equation}
with
\begin{equation}\label{eq_force2}
\frac{\partial \Psi}{\partial \mathbf{F}} = \mathbf{P} =  \mu(\mathbf{F} - \mathbf{F^{-T}}) + \lambda \ln(J)\mathbf{F^{-T}}
\end{equation}
Further derivations give the direct relationship between force
$\mathbf{f}_i$ and the current geometry:
\begin{equation}\label{eq_holly}
(\mathbf{f_2}, \mathbf{f}_3, \mathbf{f}_4) = -V_0\mathbf{P}\mathbf{D}_m^{-1}
\end{equation}
\begin{equation}\label{eq_holly2}
\mathbf{f}_1 = - (\mathbf{f}_2 + \mathbf{f}_3 + \mathbf{f}_4).
\end{equation}

Each frame, force can be calculated over elements then summed for each
node over neighboring elements:
\begin{equation}\label{eq_f}
\mathbf{f_i} = \sum_{e_j = e_1}^{e_t}f_i^{e_j}.
\end{equation}
The mass of each node is calculated by averaging the mass over
connected tetrahedra:
\begin{equation}\label{eq_mass}
\mathbf{m_i} = \frac{1}{3}\sum_{e_j = e_1}^{e_t}m_{e_j}.
\end{equation}
The damping force is given as
\begin{equation}\label{eq_damp}
\mathbf{f}_i^d = -\gamma V_0(\mathbf{v}_i - \mathbf{v}_c)\quad
\hbox{with}\quad \mathbf{v}_c = \frac{1}{3}\sum_{i = 1}^{4}
\mathbf{v}_i.
\end{equation}
Large damping coefficients are used to avoid oscillations in the
centipede's body.

A semi-implicit time integration method \cite{semi} is applied at each
time step, as follows:
\begin{equation}\label{eq_semi}
(\mathbf{I} + \Delta t\gamma
V_0\mathbf{M}^{-1}\mathbf{K})\mathbf{v}^{n+1} = \mathbf{v}^n + \Delta
t\mathbf{M}^{-1}(\mathbf{f}^e + \mathbf{g} + \mathbf{f}_{\hbox{ext}})
\end{equation}
\begin{equation}\label{eq_semi2}
\mathbf{x}^{n+1} = \mathbf{x}^{n} + \Delta t\mathbf{v}^{n+1}
\end{equation}
Due to the nonlinear part in $\mathbf{P}$, an implicit solution in
position cannot be achieved in real time.

Eventually, the implementation of this scheme can be as simple as a
mass-spring-damper system, but with the energy defined over a volume.
Physical realism and ease in control are both improved. Collision
against rigid-bodies, such as the ground, can be achieved trivially by
applying a penalty force to nodes that penetrate \cite{collision}
\cite{collision2}. Figure~\ref{fvm} shows an experimentation result
with the FVM.

Unlike fish, the deformable segments of centipedes do not act as
muscles, but simply as soft body tissues that glue the rigid segments
together. However, muscle effects can be easily realized by varying
the material's Lam\'e parameters in time \cite{fvm1}.

\subsection{Rigid and Deformable Coupling}

\begin{figure}[ptb]
\centering
\includegraphics[width=3.5in]{./images/ride1.png}
\includegraphics[width=3.5in]{./images/ride2.png}
\caption{Two way coupling of Rigid and FVM based deformable bodies}%
\label{ride}
\end{figure}

Two-way coupling of rigid body and deformable body mechanics is
trivial in mass-spring-damper systems. Two-way coupling methods for
rigid bodies and FEM-based deformable bodies have also been developed
\cite{coupling}, but with high complexity in their implementation
schemes. The coupling method proposed in this paper supports FVM based
deformable bodies and is as simple to understand and implement as a
mass-spring-damper system.

To begin with, nodes on the coupling face of the deformable object
need to be fixed to the rigid body's coupling face. Then within each
frame, there are two phases for physics. In phase one, FVM is applied
on the deformable mesh to calculate forces and update the positions of
nodes (except for the fixed nodes). In phase two, the elastic forces
from the fixed nodes are applied to the rigid bodies as an external
force and affect the rigid body's position and orientation through
rigid body dynamics. Then the fixed nodes' positions are updated
according to the new rigid body's position and orientation. The
deformable part applies its influence through the attached nodes and
the rigid part applies its influence on the deformable part through
rigid body dynamics. Figure~\ref{ride} shows experimentation results
using the coupling method.

\subsection{Legs}

\begin{figure}[ptb]
\centering
\includegraphics[width=3.5in]{./images/leg.png}
\caption{Leg structure. A: 3D view; B: 2D view.}
\label{leg}
\end{figure}

Arthropod legs are rigid exoskeletons connected to the body's
skeleton. Thus, the legs of the virtual centipede are simulated as
rigid links and directly attached to the rigid segments of the
centipede. Figure~\ref{leg} shows the structure of the centipede leg,
which is composed of four rigid segments. The four leg segments will
remain in a plane (leg plane) during rotation. The $l_0$ segment is
fixed to the rigid body segment's bottom and can rotate by an angle
$\theta$ vertically around the $y$ axis of the rigid segment's
coordinate system. There are three degrees of freedom within the leg
plane: $\alpha$ between $l_0$ and $l_1$, $\beta$ between $l_1$ and
$l_2$, and $\gamma$ between $l_2$ and $l_3$. All the angles are
calibrated with respect to the rigid segment's local coordinate
system. During simulation, the transformation matrix of the rigid
segment is used directly to place these angles correctly. Only three
degrees of freedom are needed for a unique convex solution of the leg
configuration in 3D space; thus, $\gamma$ is fixed in this model. For
each tip-root vector $\mathbf{PO}$ in 3D space, no more than one valid
convex solution of the three leg rotations ($\theta$, $\alpha$,
$\beta$) exists. More details can be found later on the inverse
kinematics of the leg.

\section{Locomotion}

\subsection{Theory}

Legged locomotion is the most common type of locomotion in terrestrial
creatures. The advantage is obvious---combinations of rotations
between different leg segments as well as different legs allows large
freedom in the positioning of the body, allowing the body to travel
over irregular terrains stably. However, at any moment the body must
be balanced or at least quasi-balanced in gravity.

Legged locomotion can be solved globally as a constrained optimization
problem. This indeed is the approach in early robotic designs
\cite{leg,centralized}. However, people soon discovered that this
approach requires too much computation for an insect's simple neural
systems \cite{walknet1}. The WALKNET \cite{walknet1,walknet2} proposed
by Cruse, et al., demonstrated a simple local leg controller that can
sense and react on its own to resolve the complex locomotion of a
six-legged insect. Robotics researchers soon proposed similar designs
and applied them onto quadruped \cite{quadruped,bigdog} and hexapod
\cite{quadruped} robots. Robotics researchers in Japan recently
proposed decentralized designs of a centipede robot's locomotion
system \cite{japan,japan2,fcp}. However, these designs in robotics do
not arise from the correct interplay of biological neural systems and
physics. Here, I propose a theory of how myriapoda species' locomotion
system works in a decentralized way.

First, consider the following thought experiment: Imagine a line of
students forming a human centipede (they are not glued together). The
first student (head) of the human centipede decides where he wants to
go and asks his subsequent student to follow him. The subsequent
student does the same thing by asking the student behind him to follow
his steps. This same rule propagates to the last student. When they
start to move, we can easily observe from above that the entire line
of students walk as if they were a whole. Three key observations
result from this thought experiment: First, the head student makes all
the locomotion decisions and implements these decisions with his own
two legs; second, the other students have only one locomotion
target---that is, to minimize the distance between the student in
front of them; third, the previous two simple local rules will result
in a global behavior.

In the above thought experiment, balance is not an issue since the
students are not connected together physically. Now imagine that they
are connected together and that their separation is always 0. The head
student still does whatever he wants to do, but the task for the
trailing students now becomes the minimization of the dragging force
(so that they feel more comfortable). At the beginning, the human
centipede might not be able to correctly behave since the students do
not respond correctly or quickly to the frontal dragging force.
However, after some training, we can expect that each student to be
able to response immediately and minimize the dragging force on the
fly.

The model used for the virtual centipede follows this idea of head
leading and local following. At any moment, the head segment
synthesizes information sensed from the antennas to make a decision in
locomotion, then this decision is processed by the target network of
the first rigid segment. The target net will output a desired future
configuration of this rigid segment---namely, the orientation and
position of the rigid segment in the next moment. Then the desired
target is handed over to the local leg controllers. The local leg
controllers will solve the desired leg rotations of the left and right
legs through inverse kinematics and inform the motor neurons to make
the change. For the subsequent segments, sensing and minimizing the
deformation force of the frontal soft segment is its target net's
(Figure~\ref{target}) task. The target net thus can be understood as
being able to interpret the physics of the deformable object. Such a
target net takes the deformation force and current configuration of
the rigid segment as input and will output a future configuration of
the rigid segment such that the deformation force in the $x-z$ plane
(Figure~\ref{leg}) of the object is minimized. \cite{emulate}
describes the neural net approach of emulating physics through
supervised learning. Figure~\ref{target} shows the diagram of the
neural net set up. A balance net is needed to balance the rigid
segment in terms of its height and orientation. The target is to keep
the height close to an objective height and minimize the unbalanced
rotation of the rigid segment. The design of balance net can be a
negative feedback network.

\subsection{A Working Local Leg Controller}

\begin{figure}[ptb]
\centering
\includegraphics[width=3.5in] {./images/targetnet.png}
\caption{The target net of a rigid segment tries to minimize the
sensed elastic force in the $x$-$z$ plane. The target net of the head
segment is different: The head segment's target net does not aim at
minimizing the elastic force, but has direct objectives such as
turning and acceleration. The head's objective net obtains its input
from higher-level neural nets in the brain.}
\label{target}
\end{figure}

\subsubsection{Target Net}

During stance mode, the target net outputs a target orientation and
position of the rigid segment to the leg controller. The target net is
homogenous for all body segments but different for the head segment.
The head segment's target net is connected to higher-level networks
and executes locomotion commands such as {\it turn-left}, {\it
turn-right}, {\it accelerate}, and {\it decelerate}. The body
segment's target net aims at minimizing the deformation of the soft
segment in front of it. A well trained target net of a body segment
can react quickly to small deformations and correctly output a target
move of the rigid segment. The resulting target net will emulate the
physics of the rigid body as well as the deformable body, since in an
unconstrained state, real physics also tries to minimize the
deformation.

When the leg is in swing forward mode, the rigid segment is completely
controlled by the deformable segments connected to it through the
physical coupling. During this mode, the root position of the leg is
updated together with the rigid segment's configuration. The target
net will generate target tip positions for the leg and the inverse
kinematics net will resolve the rotations. There are two separate
phases during the swing forward mode and two target tip positions will
be generated during this mode. One target tip position is to lift the
leg up and the other is to land the tip to the next stance position.
Reference \cite{fcp} described the ``Follow-the-Contact-Point'' model
for targeting the tip landing position.

During the first phase of the swing backward mode, the target net will
generate a target leg tip position to increase the chance of landing
on the ground. After the first phase, the target net will guide the
leg tip to the Posterior Extreme Position.

\subsubsection{Balance Net}

The target net of body segments does not guarantee the balance of the
rigid segment. The purpose of a balance net is to use a negative
feedback network to maintain balance of the body orientation as well
as its height. A simple negative feedback loop will suffice to
maintain the height of the rigid segment around an ideal height and
maintain the rigid segment's $x$-$z$ plane parallel to the local
ground. The balance net does not produce any motor movement, it only
outputs the desired position and orientation of the rigid segment.
Essentially, it can be viewed as part of the target net and it
modifies the output of the target net such that balance is achieved.

\subsubsection{Switch Net}

The leg controller net takes output from the target net and tries to
achieve the targeted rigid segment configuration by modifying its leg
rotations. This mapping from the desired leg root/tip position to leg
rotations is achieved through an inverse kinematics network. When the
leg is in stance mode, it will have its tip fixed on the ground. Then,
the desired root position of the leg is calculated from the desired
rigid segment position and orientation. If the inverse kinematics
network successfully returns a convex solution to the desired future
root position, then the motor neurons on the leg joints will achieve
the outputted rotations. If there is no solution, it means that the
leg is stretched to its extreme and cannot stretch any further. Our
current design will let the leg enter the swing forward mode when such
an extreme stretch occurs. Biologically, there are two extreme
positions---the Anterior Extreme Position (AEP) and the Posterior
Extreme Position (PEP). These positions are the critical positions
when $\theta$ (Figure~\ref{leg}) cannot increase or decrease any
further. When the PEP is detected, the leg enters the Swing Forward
state and is controlled by the Swing Forward Net. In this state, the
leg tip is released from the ground and swings forward in the air.
When the AEP is hit, the leg enters the Swing Backward state and tries
to touch the ground and enter Stance Mode again.

The Switch Net handles switching between the three leg states as well
as handing over the leg motors to the three different control nets.

\subsubsection{Stance Net}

The centipede will enter Stance Mode when its leg touches the ground.
During the Stance Mode, leg tip is fixed on the ground while the root
of the leg tries to move the rigid segment. The desired future
position and orientation of the rigid segment is determined by the
Target Net. The Inverse Kinematics Net will solve the rotations. The
Stance Net will take these rotations and send them to the motor
neurons to achieve them physically. When the Inverse Kinematics Net
cannot resolve, the Switch Net will switch the leg to the Swing
Forward state and let the Swing Froward Net take control of the motor
neurons. Also, when AEP/PEP are achieved, the leg will exit the Stance
Net and enter the Swing Backward/Forward Net.

\subsubsection{Swing Forward Net}

During a normal locomotion, a leg will alternate between the Swing
Forward state and the Stance state. The purpose of the Swing Forward
state is to move the leg tip to a new position on the ground and begin
another stance period. It is important for the Swing Forward state to
be fast, since the body segment is unsupported during this period.
Because balance is achieved through the coordination of two legs, the
Balance Net will be inactive during this state, the rigid body moves
as the deformable parts apply force and torque on it.

There are two phases during the swing forward period. The first phase
is to quickly raise the leg tip such that it leaves the ground. In my
Design, the target net will generate a target tip position that uses
the leg root's height ($y$ coordinate) as the target height, and
remain the current tip's $(x,z)$ coordinate. In the second phase, the
target net will use the AEP's tip position as the target tip position.
Reference \cite{fcp} described the ``Follow-the-Contact-Point'' model
for targeting the tip landing position, which might work even better,
but it is not biologically plausible since the leg does not have a
vision system.

\subsubsection{Swing Backward Net}

When the centipede is hung up in the air with its head and tail fixed,
the centipede's leg will alternate between the Swing Forward state and
the Swing Backward state. During the Swing Backward state, the leg tip
will first try to move down and backward to touch the ground. Then, if
no ground is touched, the leg tip will move to the PEP. The Inverse
Kinematics Net is used to generate leg rotations for a given tip
target. The Target Net is used to generate the two target tip
positions.

The Swing Backward Net will always terminate early and enter the
Stance Net during normal locomotion. It comes into play when the leg
tip fails to touch the ground as expected before the end of swing
forward state. This makes the leg adaptive on irregular terrain.

\subsection{Inverse Kinematics Net}

Within the leg controller, the inverse kinematics is the most
important module. In nature, this function is carried out implicitly
through a neural network. However, since the problem can be explicitly
solved, we did not use the neural network approach. The Inverse
Kinematics Net used in this paper take the position of the leg tip $P$
and leg root $O$, and returns $\theta$, $\alpha$, and $\beta$. If no
convex solution exists, it will inform the Switch Net to switch to the
Swing Forward state. Here, a convex solution means $\beta > 0$. The
steps to solve doe $\theta$, $\alpha$, and $\beta$ are as follows:
\begin{itemize}
\item $\gamma$ is fixed since only three degrees of freedom exist in
3D space.
\item By restricting all the leg segments in a plane, $\theta$ is
independently solved for from $\alpha$ and $\beta$. As
Figure~\ref{leg}(A) shows, $\theta$ can be solved directly using the
position of P and O.
\item After $\theta$ is obtained, $\alpha$ and $\beta$ can be solved
for by solving the position of point M. Point M is determined by
rotating segment $l_1$ around point $O$ and segments $l_2$-$l_3$
($\gamma$ is fixed) around point P. There will be no intersection (no
solution), one intersection (tangent), or two intersection of the two
circles. The convex solution is the one with $\beta > 0$.
\item If a convex solution for point M exist in the leg plane,
$\alpha$ and $\beta$ can be determined using simple geometric
calculations.
\end{itemize}

\section{Higher Level Behaviors}

\subsection{Perception Through the Antennas}

The virtual centipede uses its virtual antennas to help it survive in
its virtual world. The antennas of the centipede have sensor neurons
that can sense physical contacts, food, and hormones. This information
can be used by the higher-level networks in the centipede brain to
trigger appropriate behaviors, such as avoiding obstacles, sexual
approaches, chasing prey, etc.

\subsection{Obstacle Avoidance}

Obstacle avoidance for artificial animals can be as complicated as the
vision problem \cite{eyepaper,fish}. In our simple model, the
centipede only uses its antennas to sense physical contact. Similar to
the Braitenberg Vehicle, the centipede will turn in the direction
opposite to the antenna that senses contact. For example, if the
centipede's left antenna touches something undesirable, the locomotion
system will be informed to turn right to avoid it. The decision is
always a local one without any big picture in the centipede's mind.

\subsection{Foraging and Predating}

When searching for food, the centipede enters a random walk mode. The
brain will randomly set a short term locomotion target, such as
turning 15 degrees right, and walk 100 millimeters. Once the previous
locomotion target is achieved, a new randomly generated target will be
executed. When the molecules or hormone level of the food/prey become
intense enough for an antenna to fire a signal of detection to the
brain, the centipede will walk in a circle to calibrate the direction
of the hormone using its antenna as a gradient detector. The centipede
will follow the guess and run toward the food. Of course, there is a
chances that the centipede will make a wrong judgement.

\section{Conclusions}

We have developed a realistic computer graphics simulation model of a
centipede. The Finite Volume Method was employed to simulated
deformable segments in the centipede's body. To achieve even more
realistic biomechanics, rigid body simulation was added to the
centipede's biomechanical structure. A simple and intuitive two-way
rigid-deformable coupling method was also developed for the purposes
of centipede simulation.

A theory of decentralized motor control in myriapoda species is
proposed and a novel controller design is successfully implemented
based on this theory. The controller network works robustly to balance
the body, walk over irregular terrain, and, most interestingly,
generate beautiful leg movement waves that emerge spontaneously
without the use of a global pattern generator.

The physics-based simulation of a centipede with 30 deformable
segments can run at a rate of 2000 frames/sec on a 2009 Macbook Pro
laptop computer.


\begin{thebibliography}{9}

\bibitem{opengl} D Shreiner, OpenGL programming guide: the official guide to learning OpenGL, versions 3.0 (2010)

\bibitem{glsl}RJ Rost, OpenGL (R) shading language (2005)

\bibitem{eigen}Guennebaud and Benot Jacob and others, Eigen v3, \href{http://eigen.tuxfamily.org}{http://eigen.tuxfamily.org}


\bibitem {vehicle}Braitenberg, V. Vehicles: Experiments in synthetic
psychology. Cambridge, MA: MIT Press. (1984)

\bibitem {sixLeg}Grieco, J.C.; Prieto, M.; Armada, M.; Gonzalez de Santos, P.;
, "A six-legged climbing robot for high payloads," Control Applications, 1998.
Proceedings of the 1998 IEEE International Conference on , vol.1, no.,
pp.446-450 vol.1, 1-4 Sep (1998)

\bibitem{randall}Randall D.Beer, Intelligence as Adaptive Behavior: an experiment in computational
neuroethology, Perspecitves in Artificial Intelligence, vol.6 (1990)

\bibitem {ant}Dorigo, M.; Maniezzo, V.; Colorni, A.; , "Ant system:
optimization by a colony of cooperating agents," Systems, Man, and
Cybernetics, Part B: Cybernetics, IEEE Transactions on , vol.26, no.1,
pp.29-41, Feb (1996)

\bibitem {walknet1} H.Cruse, et al. A Biologically Inspired Controller for Hexapod
Walking: Simple Solutions by Exploiting Physical Properties. Biol. Bull. 200: 195�200. (April 2001)

\bibitem {walknet2} H.Cruse, et al. Walknet�a biologically inspired network to control six-legged walking.
Neural Networks 11 (1998)

\bibitem {karl} Karl Sims, Evolving 3D Morphology and Behavior by Competition.  Artificial life. (1994)

\bibitem {fish}Xiaoyuan Tu "Artificial Animals for Computer Animation:
Biomechanics, Locomotion, Perception, and Behavior" Ph.D Dissertation ,
Department of Computer Science, University of Toronto, January (1996)

\bibitem {salamander}Auke Jan Ijspeert, Alessandro Crespi, Dimitri Ryczko, and
Jean-Marie Cabelguen "From Swimming to Walking with a Salamander Robot Driven
by a Spinal Cord Model", Science 9 March 2007: 315 (5817), 1416-1420. (2007)

\bibitem {eyepaper}Salomon, R.; Lichtensteiger, L.; , "Exploring different
coding schemes for the evolution of an artificial insect eye," Combinations of
Evolutionary Computation and Neural Networks, 2000 IEEE Symposium on , vol.,
no., pp.10-16, (2000)

\bibitem {rigid} James K. Hahn, Realistic animation of rigid bodies, SIGGRAPH (1988)

\bibitem {rigid1}David Baraff,  An Introduction to Physically Based Modeling, rigid body simulation - unconstrained rigid body dynamics. SIGGRAPH Course Notes. (1997)

\bibitem {rigid2}David Baraff,  An Introduction to Physically Based Modeling, rigid body simulation - Nonpenetration Constraints. SIGGRAPH Course Notes. (1997)

\bibitem{coupling} Tamar Shinar, Craig Schroeder, Ronald Fedkiw, Two-way Coupling of Rigid and Deformable Bodies, Eurographics/ACM SIGGRAPH. (2008)

\bibitem{fvm1} Joseph Teran, R.Fedkiw, et al. Finite Volume Methods for the Simulation of Skeletal Muscle, Eurographics/ACM SIGGRAPH. (2003)

\bibitem{fvm2} J. Teran, R.Fedkiw, et al. Creating and Simulating Skeletal Muscle from the Visible Human Data Set. IEEE Transactions on Visualization and Computer Graphics, 11, pp. 317-328, (2005)

\bibitem{fvm3} E. Caramana and M. Shashkov. Elimination of artificial grid distortion and hourglass-type motions by means of lagrangian subzonal masses and pressures. J. Comput. Phys., (142):521�561, (1998)

\bibitem{fleshfem} J.P.Gourret, N.Magnenat-Thalmann, D.Thalmann, Simulation of Object and Human Skin Deformations in a Grasping Task, Proc.SIGGRAPH (1989)

\bibitem{skinfem} E. Sifakis, J. Hellrung, J. Teran, A. Oliker, C. Cutting. Local Flaps: A Real-Time Finite Element Based Solution to the Plastic Surgery Defect Puzzle, Studies in Health and Technology Informatics, 142, pp. 313-138, (2009)

\bibitem{bonefem} J.H. Keyak, J.M. Meagher, H.B. Skinner,  C.D. Mote Jr. Automated three-dimensional finite element modelling of bone: a new method. Journal of Biomedical Engineering, (1990)

\bibitem{musclefem} David T. Chen, David Zeltzer,
Pump it up: computer animation of a biomechanically based model of muscle using the finite element method. SIGGRAPH (1992)

\bibitem{collision} Teschner, M., Kimmerle, S., Zachmann, G., Heidelberger,
B., Raghupathi, L., Fuhrmann, A., Cani, M.P., Faure, F., Magnetat-Thalmann, N., And Strasser,
W., 2004. Collision detection for deformable objects.

\bibitem{collision2}Robert Bridson, Ronald Fedkiw, Robust treatment of collisions, contact and friction for cloth animation, SIGGRAPH (2002)

\bibitem{spring1} Terzopoulos, D., Platt, J., Barr, A., And Fleischer, K.
1987. Elastically deformable models. In Proceedings of the 14th
annual conference on Computer graphics and interactive techniques,
ACM, New York, NY, USA, SIGGRAPH , 205�214, (19787).

\bibitem{spring2} Terzopoulos, D., And Fleischer, K. Modeling inelastic
deformation: viscolelasticity, plasticity, fracture. In Proceedings
of the 15th annual conference on Computer graphics and
interactive techniques, ACM, New York, NY, USA, SIGGRAPH.(1988)

\bibitem{nonlinear} Bonet, J., And Wood, R. D. Nonlinear continuum mechanics
for finite element analysis. Communications in Numerical
Methods in Engineering 24, 11, 1567�1568. (2008)

\bibitem{numerics} Uri M. Ascher, Computer Methods for Ordinary Differential Equations and Differential-Algebraic Equations

\bibitem{semi}David Baraff, Andrew Witkin, Large Steps in Cloth Simulation. SIGGRAPH (1998)

\bibitem{leg} MH Raibert, Legged robots that balance, MIT Press(1996)

\bibitem{centralized}Martin Golubitsky, Ian Stewart, Pietro-Luciano Buono, J.J. Collins, A modular network for leggedlocomotion, Physica D: Nonlinear Phenomena (1998)

\bibitem{quadruped} Yasuhiro Fukuoka, et al. Adaptive Dynamic Walking of a Quadruped Robot on Irregular Terrain Based on Biological Concepts, Journal of Robotics Research. (2003)

\bibitem{hexapod} Uluc Saranli, et al. RHex: A Simple and Highly Mobile Hexapod Robot. Journal of Robotics Research. (2001)

\bibitem{kinematics}Marc H. Raibert, Animation of dynamic legged locomotion, MIT Leg Laboratory, SIGGRAPH (1991)

\bibitem{bigdog}M Raibert, K Blankespoor, et al. Bigdog, the rough-terrain quadruped robot, Proceedings of the 17th World Congress
The International Federation of Automatic Control
Seoul, Korea (2008)

\bibitem{japan}T.Odashima, H.Yuasa, M.Ito. The autonomous
decentralized myriapod locomotion robot which is
consist of homogeneous subsystems,JRSJ, Vol.16 (1998)

\bibitem{japan2}Shinkichi Inagaki, Hideo Yuasa, Tamio Arai.
CPG model for autonomous decentralized multi-legged
robot system.generation and transition of oscillation
patterns and dynamics of oscillators, Robotics and Autonomous Systems 44 (2003)

\bibitem{fcp}Tomoya Niwa, Shinkichi Inagaki and Tatsuya Suzuki, Locomotion Control of Multi-Legged Robot based on
Follow-the-Contact-Point Gait.ICROS-SICE International Joint Conference (2009)

\bibitem{ann}B.Yegnanarayana, Book, Artificial Neural Networks. Prentice-Hall of Inida (2006)

\bibitem{nofem1}Advances in Artificial Life: 10th European Conference, Ecal , Part 1, page 100 (2009)

\bibitem{mesh1}K Ho-Le, Finite element mesh generation methods: a review and classification, Computer Aided Design (1984)

\bibitem{mesh2}JC Caendish, DA Field, et al. An apporach to automatic three dimensional finite element mesh generation, International Journal for Numerical Methods in Engineering. (2005)

\bibitem{emulate}Radek Grzeszczuk, NeuroAnimator: Fast neural network emulation and control of physics-based models. PHD thesis (1998)

\bibitem{nofem2} \href{http://geb.uma.es/living-matter/spring-mass-systems-in-artificial-life}{$http://geb.uma.es/living-matter/spring-mass-systems-in-artificial-life$}

\bibitem{wikiALife} \href{http://en.wikipedia.org/wiki/Artificial_life}{$http://en.wikipedia.org/wiki/Artificial_life$}

\bibitem{von}\href{http://www.crystalinks.com/alife.html}{$http://www.crystalinks.com/alife.html$}

\bibitem{myriapoda}\href{http://en.wikipedia.org/wiki/Myriapoda}{$http://en.wikipedia.org/wiki/Myriapoda$}

\end{thebibliography}

\end{document} 